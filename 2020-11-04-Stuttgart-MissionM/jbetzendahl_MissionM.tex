\documentclass[aspectratio=169,xcolor=dvipsnames]{beamer}
\usetheme{Pittsburgh}
\usepackage{xcolor}
\usepackage[utf8]{inputenc}
\usepackage[german]{babel}
\usepackage{amsmath}
\usepackage{amsfonts}
\usepackage{amssymb}
\usepackage{graphicx}
\usepackage{multicol}
\usepackage{wrapfig}
\usepackage{hyperref}
\usepackage{tikz}
\usepackage{enumitem}
\usepackage{xcolor}
\usepackage{scalerel,xparse}

\usetikzlibrary{shapes,arrows,chains}

\author{Jonas Betzendahl}
\title{Meine Kollegin -- Ein Roboter?}

\beamertemplatenavigationsymbolsempty 
\setbeamercolor{frametitle}{fg=black}

%src: https://tex.stackexchange.com/questions/34921/how-to-overlap-images-in-a-beamer-slide
\def\Put(#1,#2)#3{\leavevmode\makebox(0,0){\put(#1,#2){#3}}}

\begin{document}

\NewDocumentCommand\emojieu{}{
    \scalerel*{
        \includegraphics{images/flag-european-union.png}
    }{X}
}

%------------------------------------------------------------------------------------
\section{Einführung}
\usebackgroundtemplate{\includegraphics[height=\paperheight,width=\paperwidth]{images/background_title}}

\begin{frame}
\begin{center}
\vfill
\huge Meine Kollegin -- Ein Roboter?
\normalsize 
\smallskip
\smallskip

Künstliche Intelligenz in Theorie und Praxis
\bigskip\bigskip

\large Jonas Betzendahl, M.Sc.\\\normalsize FAU Erlangen-Nürnberg
\bigskip\bigskip\large

\href{https://twitter.com/lambdatotoro}{\includegraphics[scale=0.125]{images/twitter_logo.png}}
\href{https://chaos.social/@lambdatotoro}{\includegraphics[scale=0.125]{images/mastodon_logo.png}}
\href{https://github.com/lambdaTotoro}{\includegraphics[scale=0.125]{images/github_logo.png}}

\texttt{@lambdaTotoro (@chaos.social)}
\end{center}
\end{frame}

%------------------------------------------------------------------------------------
\usebackgroundtemplate{\includegraphics[height=\paperheight,width=\paperwidth]{images/background_blank}}

\begin{frame}
\frametitle{}
\end{frame}

\begin{frame}
\frametitle{}
\end{frame}

\section{Grundlagen}

\begin{frame}
\frametitle{\glqq Algorithmen\grqq}
\begin{center}
\includegraphics[height=0.75\paperheight,keepaspectratio]{images/man-cooking-clipart} 
\end{center}
\end{frame}

\begin{frame}
\frametitle{\glqq Maschinelles Lernen\grqq}
\begin{center}
\includegraphics[height=0.7\paperheight,keepaspectratio]{images/funnel} 
\end{center}
\end{frame}

\begin{frame}
\frametitle{\glqq Künstliche Intelligenz\grqq}
\pause

\begin{minipage}{0.5\paperwidth}
\begin{center}
\includegraphics[height=0.7\paperheight,keepaspectratio]{images/man-shrug} 
\end{center}
\end{minipage}\begin{minipage}{0.5\paperwidth}

\end{minipage}
\end{frame}

%----------------------------------------------

\begin{frame}
\frametitle{Mögliche Anwendungen:}
\end{frame}

\section{Fallstudien}

\begin{frame}
\frametitle{}
\end{frame}

\begin{frame}
\frametitle{}
\end{frame}

\begin{frame}
\frametitle{}
\end{frame}

\begin{frame}
\frametitle{}
\end{frame}

\section{Richtlinien}

\begin{frame}[fragile]
\frametitle{\emph{8 Prinzipien}}
\begin{center}
Das Institut für ethisches Maschine Learning schlägt folgende Richtlinien vor:
\end{center}
\medskip

\large
\setlength{\leftmargini}{150pt}
\begin{itemize}[label=\textcolor{RedOrange}{\textbullet}]
\item Human Augmentation
\item Bias Evaluation
\item Explainability by Justification 
\item Reproducible Operations
\item Displacement Strategy
\item Practical Accuracy
\item Trust by Privacy
\item Data Risk Awareness
\end{itemize}
\end{frame}

\begin{frame}
\frametitle{\emojieu EU-Richtlinien (1)}
\end{frame}

\begin{frame}
\frametitle{Quellen}
\small
\begin{center}
\textbf{Diese Präsentation:}\\
\url{github.com/lambdaTotoro/Talks/blob/master/2020-11-04-Stuttgart-MissionM/}
\end{center}
\bigskip

\begin{itemize}
\item The Institute for Ethical AI \& Machine Learning: \url{https://ethical.institute/}
\item Awful AI: \url{https://github.com/daviddao/awful-ai}
\end{itemize}
\end{frame}

\end{document}

