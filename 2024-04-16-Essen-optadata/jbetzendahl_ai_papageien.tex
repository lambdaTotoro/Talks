\documentclass[aspectratio=169,usenames,dvipsnames]{beamer}
\usetheme{Pittsburgh}
\usepackage{xcolor}
\usepackage[utf8]{inputenc}
\usepackage[german]{babel}
\usepackage{amsmath}
\usepackage{amsfonts}
\usepackage{amssymb}
\usepackage{graphicx}
\usepackage{multicol}
\usepackage{wrapfig}
\usepackage{hyperref}
\usepackage{pdfrender}

\author{Jonas Betzendahl}
\title{Papageien am Steuer}

\beamertemplatenavigationsymbolsempty 

%src: https://tex.stackexchange.com/questions/34921/how-to-overlap-images-in-a-beamer-slide
\def\Put(#1,#2)#3{\leavevmode\makebox(0,0){\put(#1,#2){#3}}}

\definecolor{TitleColour}{rgb}{1,1,0}
\definecolor{lightestgray}{rgb}{0.95,0.95,0.95}

\begin{document}
\sffamily

%------------------------------------------------------------------------------------
\section{Introduction}

{
    \usebackgroundtemplate{\includegraphics[height=\paperheight,width=\paperwidth]{images/parrot_titlecard}}
    \setbeamertemplate{navigation symbols}{}
    \begin{frame}[fragile]
    \Put(-20,160){\textpdfrender{
		TextRenderingMode=FillStroke,
		LineWidth=.2pt,
		FillColor=TitleColour,
	}{\resizebox{0.75\linewidth}{!}{Papageien am Steuer!}}
    }
    
    \Put(-20,130){\textpdfrender{
		TextRenderingMode=FillStroke,
		LineWidth=.1pt,
		FillColor=TitleColour,
	}{\resizebox{0.85\linewidth}{!}{Risiken und Nebenwirkungen von Künstlicher Intelligenz}}
    }
    
    \Put(-20,100){
        \large
	\textcolor{yellow}{Jonas Betzendahl, M.Sc.}
    }
    \Put(-23,70){
	\textcolor{yellow}{FAU Erlangen - Nürnberg}
    }
    \Put(300,-180){
	\href{https://github.com/lambdaTotoro}{\includegraphics[scale=0.125]{images/github_logo.png}}
	\href{https://chaos.social/@lambdatotoro}{\includegraphics[scale=0.125]{images/mastodon_logo.png}}
    }
    \Put(230,-220){
	\textcolor{white}{\texttt{@LambdaTotoro (@chaos.social)}}
    }
    \end{frame}
}

\setbeamercolor{background canvas}{bg=lightestgray}

%------------------------------------------------------------------------------------
%------------------------------------------------------------------------------------
\section{Motivation}
\begin{frame}
\begin{center}
\Large
Teil 0:
\bigskip

\huge
\emph{Warum ich die Geister rief}
\end{center}
\end{frame}


\begin{frame}
\frametitle{Menschen sind \emph{sehr} fehlbar...}
\begin{minipage}{0.5\textwidth}
\begin{center}
\includegraphics[keepaspectratio, height=0.75\textheight]{images/calipers_transparent}
\end{center}
\end{minipage}\begin{minipage}{0.5\textwidth}
\begin{center}
\pause
\includegraphics[keepaspectratio, height=0.75\textheight]{images/human_bias}
\end{center}
\end{minipage}
\end{frame}

{
    \usebackgroundtemplate{\includegraphics[height=\paperheight,width=\paperwidth]{images/sloth}}
    \setbeamertemplate{navigation symbols}{}
    \begin{frame}[plain]
    \Put(150,170){\color{black}\huge \rotatebox{-10}{...und \emph{sehr} langsam!}}
    \end{frame}
}

{
    \usebackgroundtemplate{\includegraphics[height=\paperheight,width=\paperwidth]{images/elektronengehirn}}

\begin{frame}

\end{frame}
}

\section{ChatGPT}
\begin{frame}
\begin{center}
\Large
Teil I:
\bigskip

\huge
\emph{Welche Geister ich rief}\\
\Large
\emph{Tschätt Jeepy Wer\dots?}
\end{center}
\end{frame}

\begin{frame}
\begin{minipage}{0.45\textwidth}
\Put(50, 5){\includegraphics[width=0.5\textwidth, keepaspectratio]{images/OpenAI_Logo}}
\vspace*{20px}
\begin{itemize}
\item Fr\"uher Non-Profit, heute (auch) For-Profit
\item Mission: ``...to ensure that AGI benefits all of humanity...''
\pause
\end{itemize}
\end{minipage}%
\begin{minipage}{0.55\textwidth}
\begin{center}
\Large
ChatGPT\normalsize
\end{center}
\medskip

\begin{itemize}
\item ChatBot-Modell: Text rein $\rightarrow$ Text raus
\item Buzzwords:\\
Transformer-based Large Language Model
\item Trainiert auf großen Teilen des Internets:
\end{itemize}
\end{minipage}
\pause
\Put(210, -30){$\cdot$ Wikipedia,}
\pause
\Put(206, -60){$\cdot$ Internet Archive,}
\pause
\Put(202, -90){$\cdot$ Social Media (Twitter, Reddit, \dots),}
\Put(199, -120){$\cdot$ Nachrichtenseiten,}
\Put(195, -150){$\cdot$ Wissenschaftliche Papiere,}
\Put(191, -180){$\cdot$ Common Crawl Corpus,}
\Put(187, -210){$\cdot$ BooksCorpus,}
\Put(183, -240){$\cdot$ Project Gutenberg,}

\end{frame}

%------------------------------------------------------------------------------------

\begin{frame}
\begin{center}
ChatGPT ist vielseitig und scheint\\
für alles eine Antwort zu haben!

\pause
\Put(-100,-250){\includegraphics[height=0.9\textheight, keepaspectratio, angle=0]{images/backen} }

\pause
\Put(-220,0){\includegraphics[height=0.8\textheight, keepaspectratio, angle=10]{images/excel} }

\pause
\Put(-100,50){\includegraphics[height=0.7\textheight, keepaspectratio, angle=-10]{images/entschuldigung} }
\end{center}
\end{frame}

\begin{frame}
ChatGPT stören auch alberne Anfragen nicht:
\bigskip
\begin{center}
\includegraphics[width=0.9\linewidth, keepaspectratio]{images/conversation_01} 
\end{center}
\pause
\Put(300,-75){\includegraphics[width=0.3\linewidth, keepaspectratio]{images/happy_rubber_duck}}
\end{frame}

\section{Trouble in Paradise}
\begin{frame}
\begin{center}
\Large
Teil II:
\bigskip

\huge
\emph{Ärger im Paradies}
\end{center}
\end{frame}

\begin{frame}
\begin{center}
\includegraphics[width=0.9\linewidth, keepaspectratio]{images/conversation_02} 
\end{center}
\pause
\Put(350,-125){\includegraphics[width=0.3\linewidth, keepaspectratio]{images/angry_duckling}}
\end{frame}

%------------------------------------------------------------------------------------

\begin{frame}
\begin{center}
\Large
Um genauer zu verstehen, wie das passieren kann,\\
müssen wir kurz darüber reden, wie ML funktioniert.
\end{center}
\pause

\Put(250,-125){\rotatebox{35}{\Large Keine Angst!}}
\Put(250,-175){\rotatebox{35}{\Large Alles ohne Formeln!}}
\end{frame}

{
    \usebackgroundtemplate{\includegraphics[height=\paperheight,width=\paperwidth]{images/funnel}}
    \setbeamertemplate{navigation symbols}{}
    \begin{frame}[plain]
    \end{frame}
}

\begin{frame}
\begin{minipage}{0.48\textwidth}
\begin{center}
\includegraphics[width=0.8\textwidth, keepaspectratio]{images/neuron} 
\end{center}
\end{minipage}\pause\begin{minipage}{0.48\textwidth}
\begin{center}
\includegraphics[width=0.99\textwidth, keepaspectratio]{images/perceptron} 
\end{center}
\end{minipage}
\end{frame}


{
    \begin{frame}[fragile]
    \begin{center}
    \includegraphics[scale=0.275]{images/neuralnet_transparent.png} 
    \end{center}
    \pause
    \Put(-10,175){\includegraphics[width=0.2\textwidth, keepaspectratio]{images/sunflower}}
    \pause
    \Put(315,190){\includegraphics[width=0.1\textwidth, keepaspectratio]{images/thumbs-up}}
    \end{frame}
}

{
	\setbeamercolor{background canvas}{bg=LimeGreen}
    \begin{frame}[fragile]
    \begin{center}
    \includegraphics[scale=0.275]{images/neuralnet_transparent.png} 
    \end{center}
    \Put(-10,175){\includegraphics[width=0.2\textwidth, keepaspectratio]{images/sunflower}}
    \Put(315,190){\includegraphics[width=0.1\textwidth, keepaspectratio]{images/thumbs-up}}
    \end{frame}
}


{
    \begin{frame}[fragile]
    \begin{center}
    \includegraphics[scale=0.275]{images/neuralnet_transparent.png} 
    \end{center}
    \pause
    \Put(-10,175){\includegraphics[width=0.2\textwidth, keepaspectratio]{images/kangaroo}}
    \pause
    \Put(315,160){\includegraphics[width=0.1\textwidth, keepaspectratio]{images/thumbs-down}}
    \end{frame}
}


{
	\setbeamercolor{background canvas}{bg=LimeGreen}
    \begin{frame}[fragile]
    \begin{center}
    \includegraphics[scale=0.275]{images/neuralnet_transparent.png} 
    \end{center}
    \Put(-10,175){\includegraphics[width=0.2\textwidth, keepaspectratio]{images/kangaroo}}
    \Put(315,160){\includegraphics[width=0.1\textwidth, keepaspectratio]{images/thumbs-down}}
    \end{frame}
}

{
    \begin{frame}[fragile]
    \begin{center}
    \includegraphics[scale=0.275]{images/neuralnet_transparent.png} 
    \end{center}
    \pause
    \Put(-10,175){\includegraphics[width=0.2\textwidth, keepaspectratio]{images/sunflower}}
    \pause
    \Put(315,160){\includegraphics[width=0.1\textwidth, keepaspectratio]{images/thumbs-down}}
    \end{frame}
}


{
\setbeamercolor{background canvas}{bg=OrangeRed}
    \begin{frame}[fragile]
    \begin{center}
    \includegraphics[scale=0.275]{images/neuralnet_transparent.png} 
    \end{center}
    \Put(-10,175){\includegraphics[width=0.2\textwidth, keepaspectratio]{images/sunflower}}
    \Put(315,160){\includegraphics[width=0.1\textwidth, keepaspectratio]{images/thumbs-down}}
    \end{frame}
}

{
    \begin{frame}[fragile]
    \begin{center}
    \includegraphics[scale=0.275]{images/neuralnet_transparent.png} 
    \end{center}
    \Put(-10,175){\includegraphics[width=0.2\textwidth, keepaspectratio]{images/sunflower}}
    \Put(315,190){\includegraphics[width=0.1\textwidth, keepaspectratio]{images/thumbs-up}}
    \end{frame}
}

{
    \begin{frame}[fragile]
    \begin{center}
    \includegraphics[scale=0.275]{images/neuralnet_transparent.png} 
    \end{center}
    \Put(-10,175){\includegraphics[width=0.2\textwidth, keepaspectratio]{images/sunflower_1}}
    \Put(315,190){\includegraphics[width=0.1\textwidth, keepaspectratio]{images/thumbs-up}}
    \end{frame}
}


\begin{frame}
\begin{minipage}{0.45\textwidth}
\large
Machine Learning Systeme kennen \emph{nur} den Datensatz, den wir ihnen gezeigt haben.
Sie haben keinen anderen Kontext durch Welt oder Gesellschaft, wie wir Menschen.
\bigskip

Alles, was in den Trainingsdaten drin ist, wird es kennen, aber nichts darüber hinaus.
\end{minipage}\hfill\begin{minipage}{0.5\textwidth}
\begin{center}
\pause
\includegraphics[height=0.45\textheight]{images/doomguy}
\pause

\includegraphics[height=0.45\textheight]{images/obama} 
\end{center}
\end{minipage}
\end{frame}

\begin{frame}
\begin{center}
\Large
Wie kommen wir von hier zu den Fehlern in ChatGPT?
\pause
\bigskip\bigskip

ChatGPT gibt keine \emph{echten} Antworten, sondern\\ nur Statistik gepresst in die \emph{Form} von Antworten!
\end{center}
\end{frame}

\begin{frame}
\begin{center}
\vfill
$$\qquad$$
\vfill

\Put(0, 50){\includegraphics[scale=0.6]{images/OpenAI_Logo_Single.png}}

\pause 
\Put(-90,130){\textpdfrender{
		TextRenderingMode=FillStroke,
		LineWidth=.2pt,
		FillColor=black,
	}{\resizebox{0.4\textheight}{!}{\}}}
    }
\Put(-180,230){\includegraphics[width=0.15\linewidth, keepaspectratio]{images/wikipedia_logo}}
\Put(-200,145){\includegraphics[width=0.05\linewidth, keepaspectratio]{images/shapes_circle}}
\Put(-180,145){\includegraphics[width=0.05\linewidth, keepaspectratio]{images/shapes_square}}
\Put(-160,145){\includegraphics[width=0.05\linewidth, keepaspectratio]{images/shapes_circle}}
\Put(-140,145){\includegraphics[width=0.05\linewidth, keepaspectratio]{images/shapes_square}}

\Put(-185,50){\includegraphics[width=0.05\linewidth, keepaspectratio]{images/shapes_octagon}}
\Put(-165,50){\includegraphics[width=0.05\linewidth, keepaspectratio]{images/shapes_octagon}}
\Put(-145,50){\includegraphics[width=0.05\linewidth, keepaspectratio]{images/shapes_circle}}
\Put(-185,-100){\includegraphics[width=0.13\linewidth, keepaspectratio]{images/tagesschau_logo}}
    
\pause
\Put(95,225){\includegraphics[width=0.3\linewidth, keepaspectratio]{images/speech_bubble_round}}

\pause
\Put(95,160){\includegraphics[width=0.3\linewidth, keepaspectratio, angle=180]{images/speech_bubble_square}}
\end{center}

\end{frame}

\begin{frame}
\begin{center}
\Large
Was ist die korrekte Fortsetzung für diese Reihe?
\bigskip\bigskip\bigskip

\Put(-145,50){\includegraphics[width=0.05\linewidth, keepaspectratio]{images/shapes_square}}
\Put(-115,50){\includegraphics[width=0.05\linewidth, keepaspectratio]{images/shapes_octagon}}
\Put(-85,50){\includegraphics[width=0.05\linewidth, keepaspectratio]{images/shapes_octagon}}
\Put(-55,50){\includegraphics[width=0.05\linewidth, keepaspectratio]{images/shapes_square}}
\Put(-25,50){\includegraphics[width=0.05\linewidth, keepaspectratio]{images/shapes_octagon}}
\Put(5,50){\includegraphics[width=0.05\linewidth, keepaspectratio]{images/shapes_square}}
\Put(35,50){\includegraphics[width=0.05\linewidth, keepaspectratio]{images/shapes_octagon}}
\Put(65,50){\includegraphics[width=0.05\linewidth, keepaspectratio]{images/shapes_square}}
\Put(95,50){\textpdfrender{
		TextRenderingMode=FillStroke,
		LineWidth=.1pt,
		FillColor=black,
		StrokeColor=black,
	}{\resizebox{0.1\linewidth}{!}{\dots}}
    }

\pause

Fangfrage!
\bigskip\bigskip\large

\begin{minipage}{0.33\textwidth}
\begin{center}
$\qquad$ Primzahlen
\end{center}
\end{minipage}%
\begin{minipage}{0.33\textwidth}
\begin{center}
$\qquad$ Fußballergebnisse
\end{center}
\end{minipage}%
\begin{minipage}{0.33\textwidth}
\begin{center}
$\qquad$ Ganz was anderes?
\end{center}
\end{minipage}
\bigskip\bigskip\bigskip\bigskip

Je nachdem, was die Bausteine bedeuten, kann jede Fortsetzung korrekt sein und ohne mehr Informationen können wir nicht einschätzen, ob es stimmt.
\end{center}
\Put(70,150){\includegraphics[width=0.05\linewidth, keepaspectratio]{images/shapes_square}}
\Put(200,150){\includegraphics[width=0.05\linewidth, keepaspectratio]{images/shapes_octagon}}
\Put(330,150){\includegraphics[width=0.05\linewidth, keepaspectratio]{images/shapes_triangle}}
\end{frame}

\begin{frame}
\Put(235,-425){\includegraphics[scale=0.25]{images/parrot_profile.png}}

\begin{minipage}{0.5\textwidth}

Folgender Sachverhalt ist extrem wichtig, um ChatGPT richtig einschätzen zu können:

\begin{center}
\begin{itemize}
\item Menschen benutzen Sprache, um Bedeutung und Gefühle zu vermitteln.\pause
\item ChatGPT ist ein \emph{statistisches Modell} auf Sprachbausteinen, wie Menschen sie benutzen.\pause
\item Das bedeutet \emph{nicht} (!), dass ChatGPT selbst Bedeutung und Gefühle vermitteln kann oder will. Es plappert uns nach, ohne zu verstehen.
\end{itemize}
\end{center}
\end{minipage}%
\begin{minipage}{0.5\textwidth}
\vfill
$$\quad$$
\vfill
\end{minipage}%
\end{frame}

\begin{frame}
\begin{center}
\vfill

\includegraphics[height=0.9\textheight, keepaspectratio]{images/paper.png} 
\end{center}
\Put(-85, 170){\includegraphics[width=0.45\textwidth, keepaspectratio]{images/Timnit-Gebru}}
\Put(290, 50){\includegraphics[width=0.35\textwidth, keepaspectratio, angle=5]{images/emily_bender}}
\end{frame}

\section{Parrots against Apes}
\begin{frame}
\begin{center}
\Large
Teil III:
\bigskip

\huge
\emph{Papageien gegen Affen}
\end{center}
\end{frame}

\begin{frame}
\begin{center}
\glqq Entscheidungen, die von der allgemeinen Bevölkerung über neuartige\\
Technologien getroffen werden, hängen viel stärker davon ab, was die\\
Bevölkerung diesen Technologien zuschreibt als davon, was sie\\
tatsächlich sind und tatsächlich können und nicht können.\grqq
\bigskip

Joseph Weizenbaum (1976)
\end{center}
\end{frame}

\begin{frame}
\frametitle{Binsenweisheit}
\begin{center}
\Large So stellen wir uns Tech-Trends vor\dots
\bigskip

\includegraphics[height=0.7\textheight]{hype_cycle}
\end{center}
\end{frame}

\begin{frame}
\frametitle{Verlässlichkeit}
\begin{minipage}{0.48\textwidth}
\begin{center}
\includegraphics[width=\textwidth, keepaspectratio]{images/chatgpt_hallucinate_1}
\end{center}
\pause
\end{minipage}\hfill\begin{minipage}{0.48\textwidth}
\begin{center}
\includegraphics[height=0.75\textheight, keepaspectratio]{images/marines_article}
\end{center}
\end{minipage}
\end{frame}

\begin{frame}
\frametitle{Umwelt}
\begin{minipage}{0.33\textwidth}
\large Neue KI braucht viel Hardware (mehrere Rechenzentren), Energie (im Training und im laufenden Betrieb) und Wasser (zur Kühlung) \dots
\end{minipage}\hfill\begin{minipage}{0.55\textwidth}
\begin{center}
\includegraphics[width=\textwidth]{images/carbon}
\end{center}
\end{minipage}
\end{frame}

\begin{frame}
\frametitle{Haftung}
\begin{center}
\includegraphics[height=0.8\textheight]{images/one_dollar_chevy.png} 
\end{center}
\pause
\Put(60,255){\includegraphics[height=0.75\textheight, keepaspectratio, angle=-5]{images/tesla_crash}}
\end{frame}

\begin{frame}
\frametitle{Sweatshop-Arbeit}
\begin{center}
\includegraphics[height=0.8\textheight, keepaspectratio]{images/time_kenya}
\end{center}
\end{frame}

\begin{frame}
\frametitle{Reproduzieren alter Vorurteile}
\begin{minipage}{0.45\textwidth}
\begin{center}
\includegraphics[width=0.8\textwidth, keepaspectratio]{images/amazon_hiring}
\end{center}
\pause
\end{minipage}\hfill\begin{minipage}{0.45\textwidth}
\begin{center}
\includegraphics[width=0.8\textwidth, keepaspectratio]{images/mashable}
\end{center}
\end{minipage}
\end{frame}

\begin{frame}
\frametitle{Die Zukunft} 
\begin{center}
\Large
Aber die Entwicklung geht ja weiter!\\
Was ist mit \texttt{GPT-5}, \texttt{-6} oder \texttt{-17}?! Werden die besser?
\pause\bigskip

Ja, aber\dots\medskip

\includegraphics[height=0.4\textheight]{images/ouroboros}
\end{center}
\end{frame}

%------------------------------------------------------------------------------------

\section{End}

\section{Trouble in Paradise}
\begin{frame}
\begin{center}
\Large
Teil IV:
\bigskip

\huge
\emph{\dots und wat lernt mich dat?}
\end{center}
\end{frame}

\begin{frame}
\frametitle{Verantwortung (1979)}
\Put(25,25){\includegraphics[scale=1.5, keepaspectratio]{images/accountable_management_p1}}
\Put(90,-200){\includegraphics[scale=1.5, keepaspectratio]{images/accountable_management_p2}}
\end{frame}

\begin{frame}
\begin{minipage}{0.49\textwidth}
\begin{center}
\includegraphics[width=\textwidth]{images/interaction_wrong}
\pause
\Put(0,50){\includegraphics[scale=1]{images/check_negative}} 
\pause
\end{center}
\end{minipage}\hfill\begin{minipage}{0.49\textwidth}
\begin{center}
\includegraphics[width=\textwidth]{images/interaction_correct}
\pause
\Put(-5,50){\includegraphics[scale=1]{images/check_positive}} 
\end{center}
\end{minipage}
\end{frame}

\begin{frame}

\Put(200,-467){\includegraphics[scale=0.4]{images/parrot_wing.png}}

\begin{minipage}{0.55\textwidth}
\huge
Fazit!\bigskip\large 

Am Ende des Tages gilt für moderne KI\dots
\begin{center}
\begin{itemize}
\item Es ist kein Wesen sondern ein \emph{Produkt} in den Händen einer Firma.\pause
\item Es ist begrenzt durch Trainingsdaten und macht Fehler mit Überzeugung.\pause
\item Es wird oft von Menschen überschätzt.\pause
\item \emph{Es sollte keine Verantwortung haben, die ein Papagei nicht auch übernehmen könnte.}
\end{itemize}
\end{center}
\end{minipage}%
\begin{minipage}{0.45\textwidth}
\vfill
$$\quad$$
\vfill
\end{minipage}%
\end{frame}

\begin{frame}[fragile]
\frametitle{Quellen:}
\scriptsize
\begin{center}
``Weltrettung braucht Wissenschaft''

\url{www.amazon.de/Weltrettung-braucht-Wissenschaft-Antworten-dr%C3%A4ngenden/dp/3499010062}
\end{center}
\medskip

\begin{itemize}
\item Bender, Gebru et al.: ``On the Dangers of Stochastic Parrots: Can Language Models Be Too Big?'' \url{https://dl.acm.org/doi/pdf/10.1145/3442188.3445922}
\item Dan McQuillan: ``Resisting AI: An Anti-Fascist Approach to Artificial Intelligence'' \url{https://www.jstor.org/stable/j.ctv2rcnp21}
\item Billy Perigo: ``OpenAI Used Kenyan Workers on Less Than $\$$2 Per Hour to Make ChatGPT Less Toxic'': \url{https://time.com/6247678/openai-chatgpt-kenya-workers/}
\item Scobel, ``Kulturschock durch KI'': \url{https://www.3sat.de/wissen/scobel/scobel---kulturschock-durch-ki-100.html}
\item Mutale Nkonde: ``ChatGPT: New AI system, old bias?'' \url{https://mashable.com/article/chatgpt-ai-racism-bias}
\item Richard Waters: ``Man beats machine at Go in human victory over AI'': \url{https://arstechnica.com/information-technology/2023/02/man-beats-machine-at-go-in-human-victory-over-ai/}
\item Max Hauptman: ``Marines outwitted an AI security camera by hiding in a cardboard box and pretending to be trees'' \url{https://taskandpurpose.com/news/marines-ai-paul-scharre/}
\end{itemize}
\end{frame}

\begin{frame}
\frametitle{Bonus: GO}

\begin{center}
Wir dachten auch, dass nie wieder ein Mensch gegen\\ einen guten Computer in Go gewinnt\dots\bigskip

\includegraphics[width=0.6\textwidth]{images/kejie}
\pause
\Put(-250,150){\includegraphics[height=0.8\textheight,angle=5]{images/ai_go_article}}
\end{center}
\end{frame}
\end{document}

